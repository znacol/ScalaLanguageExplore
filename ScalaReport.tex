\documentclass[12pt]{article}
\pagestyle{plain}
\usepackage[margin=25mm]{geometry}
%==========SETTINGS FOR IN-TEXT SCALA CODE=============================================
\usepackage{listings}
\usepackage{color}

\definecolor{dkgreen}{rgb}{0,0.6,0}
\definecolor{gray}{rgb}{0.5,0.5,0.5}
\definecolor{mauve}{rgb}{0.58,0,0.82}

\lstset{
  language=Java,
  aboveskip=3mm,
  belowskip=3mm,
  showstringspaces=false,
  columns=flexible,
  basicstyle={\small\ttfamily},
  numbers=left,
  numberstyle=\tiny\color{gray},
  keywordstyle=\color{blue},
  commentstyle=\color{dkgreen},
  breaklines=true,
  breakatwhitespace=true,
  tabsize=3
}
%==========DOCUMENT HEADER=============================================================
\begin{document}
\title{Scala: Language Explore}
\date{December 7, 2015}
\author{Zoe Nacol \\ Kyle Dymowski \\ Charles Tandy \\ Anthony Nguyen}
\maketitle

%==========ZOE=========================================================================
\section{Introduction to Scala}
	\paragraph{}Scala is both an object-oriented and functional language [1]. A portmanteau on ``scalable" and ``language", Scala was created based upon criticism of Java. Scala runs on the Java Virtual Machine. Classes and libraries from both Scala and Java can be freely mixed between the languages.
\section{Scala Features}
	\subsection{Concepts Learned in Class}
	\paragraph{Pattern Matching}
	\paragraph{Type Inference}
	\paragraph{Higher Order Functions}
	\paragraph{} other: functions are objects, lazy evaluation, currying
%==========KYLE=========================================================================
	\subsection{Concepts Not Learned in Class}
		\paragraph{Value Classes and Universal Traits} In Java there is no difference between the integer type, it can be used through the program for counters, an entity identifier or a number in an arithmetic expression. In most cases these Integers have nothing to do with each other. It is a common mistake to compare them do arithmetic operations on them, or pass the wrong one into a function as a parameter. A value class allows a new level of type checking that looks at the code rather than relying on the person doing the coding. A value class works like a wrapper class that gives the ability to enrich objects with a precise type without allocation overhead. 

There are a few limitations that come with using value classes, so they are cannot be universally used. 

		\begin{description}
			\item The primary constructor for the value class must have exactly one val parameter whose type is not a class[3]
			\item The class may not have specialized type parameters.
			\item The class may not have nested or local classes, traits, or objects
			\item The class may not define a equals or hashCode method.
			\item The class must be a top-level class or a member of a statically accessible object
			\item The class can only have defs as members. In particular, it cannot have lazy vError! Hyperlink reference not valid.als, vars, or vals as members.
			\item The class cannot be extended by another class.
		\end{description}

		\paragraph{Implicit Classes}Scala allows for extension methods to be added onto another type by introducing implicit classes. The main purpose of an implicit class is to make the process more concise. The method definitions must be declared where the scope allows for method definitions. This works by the implicit method mimicking the constructor of the class[4]. 

		\paragraph{String Interpolation} Scala makes use of string interpolation to create strings of data. This process is used to directly embed variables into string literals. It makes use of three methods of string interpolation. This first is s where string literal allows the usage of variables directly in the string. Second, is f which allows the creation of simple formatted strings similar to printf in PHP. Third, is raw which is essentially the same as s, but with no escaping of literals in the string[5]. 


%========ANTHONY===========================================================================
		\paragraph{Type System}This should be moved up to concepts learned in class.
		
		\paragraph{Raw Strings}

		\paragraph{Flexibility}

		\paragraph{Type Enrichment}

		\paragraph{Concurrency}

%========CHARLES===========================================================================
\section{Demo Program}
	\paragraph{}

% IF YOU WANT TO PUT CODE EXAMPLES IN DOCUMENT, looks like this:
	\begin{lstlisting}
		val height = 1.9d
		val name = "James"
		println(f"$name\%s is $height\%2.2f meters tall") // James is 1.90 meters tall
	\end{lstlisting}
\section{References}
	\begin{enumerate}
		\item
			http://www.scala-lang.org/what-is-scala.html
		\item
			https://en.wikipedia.org/wiki/Scala\_\%28programming\_language\%29
		\item
			https://ivanyu.me/blog/2014/12/14/value-classes-in-scala/
		\item 
			http://docs.scala-lang.org/overviews/core/implicit-classes.html
		\item
			http://docs.scala-lang.org/overviews/core/string-interpolation.html
	\end{enumerate}
\end{document}
