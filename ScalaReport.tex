\documentclass[12pt]{article}
\pagestyle{plain}
\usepackage[margin=25mm]{geometry}
%==========SETTINGS FOR IN-TEXT SCALA CODE=============================================
\usepackage{listings}
\usepackage{color}

\definecolor{dkgreen}{rgb}{0,0.6,0}
\definecolor{gray}{rgb}{0.5,0.5,0.5}
\definecolor{mauve}{rgb}{0.58,0,0.82}

\lstset{
  language=Java,
  aboveskip=3mm,
  belowskip=3mm,
  showstringspaces=false,
  columns=flexible,
  basicstyle={\small\ttfamily},
  numbers=left,
  numberstyle=\tiny\color{gray},
  keywordstyle=\color{blue},
  commentstyle=\color{dkgreen},
  breaklines=true,
  breakatwhitespace=true,
  tabsize=3
}
%==========DOCUMENT HEADER=============================================================
\begin{document}
\title{Scala: Language Explore}
\date{December 7, 2015}
\author{Zoe Nacol \\ Kyle Dymowski \\ Charles Tandy \\ Anthony Nguyen}
\maketitle

%==========ZOE=========================================================================
\section{Introduction to Scala}
	\paragraph{}Scala is both an object-oriented and functional language [1]. A portmanteau on ``scalable" and ``language", Scala was created based upon criticism of Java. Scala runs on the Java Virtual Machine. Classes and libraries from both Scala and Java can be freely mixed between the languages.
\section{Scala Features}
	\subsection{Concepts Learned in Class}
	\paragraph{Pattern Matching}
	\paragraph{Type Inference}
	\paragraph{Higher Order Functions}
	\paragraph{} other: functions are objects, lazy evaluation, currying
%==========KYLE=========================================================================
	\subsection{Concepts Not Learned in Class}
		\paragraph{Value Classes and Universal Traits} Paragraph text goes here.

		\paragraph{Implicit Classes}

		\paragraph{String Interpolation} This should be moved up to concepts learned in class.

%========ANTHONY===========================================================================
		\paragraph{Type System}This should be moved up to concepts learned in class.
		
		\paragraph{Raw Strings}

		\paragraph{Flexibility}

		\paragraph{Type Enrichment}

		\paragraph{Concurrency}

%========CHARLES===========================================================================
\section{Demo Program}
	\paragraph{}

% IF YOU WANT TO PUT CODE EXAMPLES IN DOCUMENT, looks like this:
	\begin{lstlisting}
		val height = 1.9d
		val name = "James"
		println(f"$name\%s is $height\%2.2f meters tall") // James is 1.90 meters tall
	\end{lstlisting}
\section{References}
	\begin{enumerate}
		\item
			http://www.scala-lang.org/what-is-scala.html
		\item
			https://en.wikipedia.org/wiki/Scala\_\%28programming\_language\%29
	\end{enumerate}
\end{document}
